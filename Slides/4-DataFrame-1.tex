% Options for packages loaded elsewhere
\PassOptionsToPackage{unicode}{hyperref}
\PassOptionsToPackage{hyphens}{url}
%
\documentclass[
]{article}
\usepackage{amsmath,amssymb}
\usepackage{iftex}
\ifPDFTeX
  \usepackage[T1]{fontenc}
  \usepackage[utf8]{inputenc}
  \usepackage{textcomp} % provide euro and other symbols
\else % if luatex or xetex
  \usepackage{unicode-math} % this also loads fontspec
  \defaultfontfeatures{Scale=MatchLowercase}
  \defaultfontfeatures[\rmfamily]{Ligatures=TeX,Scale=1}
\fi
\usepackage{lmodern}
\ifPDFTeX\else
  % xetex/luatex font selection
\fi
% Use upquote if available, for straight quotes in verbatim environments
\IfFileExists{upquote.sty}{\usepackage{upquote}}{}
\IfFileExists{microtype.sty}{% use microtype if available
  \usepackage[]{microtype}
  \UseMicrotypeSet[protrusion]{basicmath} % disable protrusion for tt fonts
}{}
\makeatletter
\@ifundefined{KOMAClassName}{% if non-KOMA class
  \IfFileExists{parskip.sty}{%
    \usepackage{parskip}
  }{% else
    \setlength{\parindent}{0pt}
    \setlength{\parskip}{6pt plus 2pt minus 1pt}}
}{% if KOMA class
  \KOMAoptions{parskip=half}}
\makeatother
\usepackage{xcolor}
\usepackage[margin=1in]{geometry}
\usepackage{color}
\usepackage{fancyvrb}
\newcommand{\VerbBar}{|}
\newcommand{\VERB}{\Verb[commandchars=\\\{\}]}
\DefineVerbatimEnvironment{Highlighting}{Verbatim}{commandchars=\\\{\}}
% Add ',fontsize=\small' for more characters per line
\usepackage{framed}
\definecolor{shadecolor}{RGB}{248,248,248}
\newenvironment{Shaded}{\begin{snugshade}}{\end{snugshade}}
\newcommand{\AlertTok}[1]{\textcolor[rgb]{0.94,0.16,0.16}{#1}}
\newcommand{\AnnotationTok}[1]{\textcolor[rgb]{0.56,0.35,0.01}{\textbf{\textit{#1}}}}
\newcommand{\AttributeTok}[1]{\textcolor[rgb]{0.13,0.29,0.53}{#1}}
\newcommand{\BaseNTok}[1]{\textcolor[rgb]{0.00,0.00,0.81}{#1}}
\newcommand{\BuiltInTok}[1]{#1}
\newcommand{\CharTok}[1]{\textcolor[rgb]{0.31,0.60,0.02}{#1}}
\newcommand{\CommentTok}[1]{\textcolor[rgb]{0.56,0.35,0.01}{\textit{#1}}}
\newcommand{\CommentVarTok}[1]{\textcolor[rgb]{0.56,0.35,0.01}{\textbf{\textit{#1}}}}
\newcommand{\ConstantTok}[1]{\textcolor[rgb]{0.56,0.35,0.01}{#1}}
\newcommand{\ControlFlowTok}[1]{\textcolor[rgb]{0.13,0.29,0.53}{\textbf{#1}}}
\newcommand{\DataTypeTok}[1]{\textcolor[rgb]{0.13,0.29,0.53}{#1}}
\newcommand{\DecValTok}[1]{\textcolor[rgb]{0.00,0.00,0.81}{#1}}
\newcommand{\DocumentationTok}[1]{\textcolor[rgb]{0.56,0.35,0.01}{\textbf{\textit{#1}}}}
\newcommand{\ErrorTok}[1]{\textcolor[rgb]{0.64,0.00,0.00}{\textbf{#1}}}
\newcommand{\ExtensionTok}[1]{#1}
\newcommand{\FloatTok}[1]{\textcolor[rgb]{0.00,0.00,0.81}{#1}}
\newcommand{\FunctionTok}[1]{\textcolor[rgb]{0.13,0.29,0.53}{\textbf{#1}}}
\newcommand{\ImportTok}[1]{#1}
\newcommand{\InformationTok}[1]{\textcolor[rgb]{0.56,0.35,0.01}{\textbf{\textit{#1}}}}
\newcommand{\KeywordTok}[1]{\textcolor[rgb]{0.13,0.29,0.53}{\textbf{#1}}}
\newcommand{\NormalTok}[1]{#1}
\newcommand{\OperatorTok}[1]{\textcolor[rgb]{0.81,0.36,0.00}{\textbf{#1}}}
\newcommand{\OtherTok}[1]{\textcolor[rgb]{0.56,0.35,0.01}{#1}}
\newcommand{\PreprocessorTok}[1]{\textcolor[rgb]{0.56,0.35,0.01}{\textit{#1}}}
\newcommand{\RegionMarkerTok}[1]{#1}
\newcommand{\SpecialCharTok}[1]{\textcolor[rgb]{0.81,0.36,0.00}{\textbf{#1}}}
\newcommand{\SpecialStringTok}[1]{\textcolor[rgb]{0.31,0.60,0.02}{#1}}
\newcommand{\StringTok}[1]{\textcolor[rgb]{0.31,0.60,0.02}{#1}}
\newcommand{\VariableTok}[1]{\textcolor[rgb]{0.00,0.00,0.00}{#1}}
\newcommand{\VerbatimStringTok}[1]{\textcolor[rgb]{0.31,0.60,0.02}{#1}}
\newcommand{\WarningTok}[1]{\textcolor[rgb]{0.56,0.35,0.01}{\textbf{\textit{#1}}}}
\usepackage{graphicx}
\makeatletter
\def\maxwidth{\ifdim\Gin@nat@width>\linewidth\linewidth\else\Gin@nat@width\fi}
\def\maxheight{\ifdim\Gin@nat@height>\textheight\textheight\else\Gin@nat@height\fi}
\makeatother
% Scale images if necessary, so that they will not overflow the page
% margins by default, and it is still possible to overwrite the defaults
% using explicit options in \includegraphics[width, height, ...]{}
\setkeys{Gin}{width=\maxwidth,height=\maxheight,keepaspectratio}
% Set default figure placement to htbp
\makeatletter
\def\fps@figure{htbp}
\makeatother
\setlength{\emergencystretch}{3em} % prevent overfull lines
\providecommand{\tightlist}{%
  \setlength{\itemsep}{0pt}\setlength{\parskip}{0pt}}
\setcounter{secnumdepth}{-\maxdimen} % remove section numbering
\ifLuaTeX
  \usepackage{selnolig}  % disable illegal ligatures
\fi
\usepackage{bookmark}
\IfFileExists{xurl.sty}{\usepackage{xurl}}{} % add URL line breaks if available
\urlstyle{same}
\hypersetup{
  pdftitle={KHUNG DỮ LIỆU (DATA FRAMES) PHẦN 1},
  pdfauthor={ThS. Lê Nhật Tùng},
  hidelinks,
  pdfcreator={LaTeX via pandoc}}

\title{KHUNG DỮ LIỆU (DATA FRAMES) PHẦN 1}
\author{ThS. Lê Nhật Tùng}
\date{}

\begin{document}
\maketitle

{
\setcounter{tocdepth}{2}
\tableofcontents
}
\section{KHUNG DỮ LIỆU (DATA
FRAME)}\label{khung-dux1eef-liux1ec7u-data-frame}

Cấu trúc được sử dụng để lưu trữ dữ liệu ở dạng bảng (cấu trúc phổ biến
nhất trong phân tích dữ liệu thống kê và học máy). Nó có thể được xem
như là một danh sách các vector có độ dài bằng nhau (thường có tên duy
nhất). Đây là cấu trúc dữ liệu cơ bản quan trọng nhất trong môi trường
tidyverse.

\subsection{TẠO KHUNG DỮ LIỆU}\label{tux1ea1o-khung-dux1eef-liux1ec7u}

Đối tượng khung dữ liệu được tạo ra bằng hàm \texttt{data.frame()}, nhận
các vector có độ dài bằng nhau làm đầu vào (có thể có các kiểu dữ liệu
khác nhau). Khung dữ liệu cũng có thể được tạo bằng cách chuyển đổi từ
ma trận với hàm \texttt{as.data.frame()}.

Các vector trong data.frame được lưu trữ theo chiều dọc và tên của các
vector gốc được lưu trữ làm tên cột. Khung dữ liệu là một lưu trữ dữ
liệu dạng bảng (giống như trong Excel).

\begin{Shaded}
\begin{Highlighting}[]
\CommentTok{\# Tạo khung dữ liệu}
\CommentTok{\# Các vector phải có độ dài bằng nhau, nhưng có thể có kiểu dữ liệu khác nhau}
\NormalTok{column1 }\OtherTok{\textless{}{-}} \FunctionTok{c}\NormalTok{(}\DecValTok{1}\SpecialCharTok{:}\DecValTok{3}\NormalTok{)}
\NormalTok{column2 }\OtherTok{\textless{}{-}} \FunctionTok{c}\NormalTok{(}\StringTok{"Anna"}\NormalTok{, }\StringTok{"Tom"}\NormalTok{, }\StringTok{"Sue"}\NormalTok{)}
\NormalTok{column3 }\OtherTok{\textless{}{-}} \FunctionTok{c}\NormalTok{(}\ConstantTok{TRUE}\NormalTok{, }\ConstantTok{TRUE}\NormalTok{, }\ConstantTok{FALSE}\NormalTok{)}

\CommentTok{\# Tạo data.frame}
\NormalTok{dataset1 }\OtherTok{\textless{}{-}} \FunctionTok{data.frame}\NormalTok{(column1, column2, column3)}
\NormalTok{dataset1}
\end{Highlighting}
\end{Shaded}

\begin{verbatim}
##   column1 column2 column3
## 1       1    Anna    TRUE
## 2       2     Tom    TRUE
## 3       3     Sue   FALSE
\end{verbatim}

\begin{Shaded}
\begin{Highlighting}[]
\CommentTok{\# Hiển thị tên cột}
\FunctionTok{colnames}\NormalTok{(dataset1)}
\end{Highlighting}
\end{Shaded}

\begin{verbatim}
## [1] "column1" "column2" "column3"
\end{verbatim}

\begin{Shaded}
\begin{Highlighting}[]
\CommentTok{\# Đổi tên cột thứ hai}
\FunctionTok{colnames}\NormalTok{(dataset1)[}\DecValTok{2}\NormalTok{] }\OtherTok{\textless{}{-}} \StringTok{"name"}
\NormalTok{dataset1}
\end{Highlighting}
\end{Shaded}

\begin{verbatim}
##   column1 name column3
## 1       1 Anna    TRUE
## 2       2  Tom    TRUE
## 3       3  Sue   FALSE
\end{verbatim}

\subsection{LẤY GIÁ TRỊ TỪ KHUNG DỮ
LIỆU}\label{lux1ea5y-giuxe1-trux1ecb-tux1eeb-khung-dux1eef-liux1ec7u}

Vector tên cần phải tương thích với chiều của dữ liệu được đặt tên.

\begin{Shaded}
\begin{Highlighting}[]
\CommentTok{\# Lấy dữ liệu từ khung dữ liệu}

\CommentTok{\# Cách 1 {-} sử dụng chỉ số như trong ma trận}
\NormalTok{dataset1[}\DecValTok{3}\NormalTok{, }\DecValTok{2}\NormalTok{]  }\CommentTok{\# hàng 3, cột 2}
\end{Highlighting}
\end{Shaded}

\begin{verbatim}
## [1] "Sue"
\end{verbatim}

\begin{Shaded}
\begin{Highlighting}[]
\CommentTok{\# Cách 2 {-} chọn dữ liệu theo tên cột}
\NormalTok{dataset1[}\StringTok{"name"}\NormalTok{]  }\CommentTok{\# toàn bộ vector tên}
\end{Highlighting}
\end{Shaded}

\begin{verbatim}
##   name
## 1 Anna
## 2  Tom
## 3  Sue
\end{verbatim}

\begin{Shaded}
\begin{Highlighting}[]
\NormalTok{dataset1[, }\StringTok{"name"}\NormalTok{]  }\CommentTok{\# ký hiệu thay thế}
\end{Highlighting}
\end{Shaded}

\begin{verbatim}
## [1] "Anna" "Tom"  "Sue"
\end{verbatim}

\begin{Shaded}
\begin{Highlighting}[]
\NormalTok{dataset1}\SpecialCharTok{$}\NormalTok{name  }\CommentTok{\# ký hiệu thuận tiện}
\end{Highlighting}
\end{Shaded}

\begin{verbatim}
## [1] "Anna" "Tom"  "Sue"
\end{verbatim}

\begin{Shaded}
\begin{Highlighting}[]
\NormalTok{dataset1[}\DecValTok{3}\NormalTok{, }\StringTok{"name"}\NormalTok{]  }\CommentTok{\# chỉ tên từ hàng thứ 3}
\end{Highlighting}
\end{Shaded}

\begin{verbatim}
## [1] "Sue"
\end{verbatim}

\begin{Shaded}
\begin{Highlighting}[]
\CommentTok{\# Cách 3 {-} sử dụng tên hàng}
\FunctionTok{rownames}\NormalTok{(dataset1) }\OtherTok{\textless{}{-}} \FunctionTok{c}\NormalTok{(}\StringTok{"girl"}\NormalTok{, }\StringTok{"boy"}\NormalTok{, }\StringTok{"teacher"}\NormalTok{)}
\NormalTok{dataset1}
\end{Highlighting}
\end{Shaded}

\begin{verbatim}
##         column1 name column3
## girl          1 Anna    TRUE
## boy           2  Tom    TRUE
## teacher       3  Sue   FALSE
\end{verbatim}

\begin{Shaded}
\begin{Highlighting}[]
\NormalTok{dataset1[}\StringTok{"teacher"}\NormalTok{, }\StringTok{"name"}\NormalTok{]}
\end{Highlighting}
\end{Shaded}

\begin{verbatim}
## [1] "Sue"
\end{verbatim}

\subsection{BỘ DỮ LIỆU CÓ
SẴN}\label{bux1ed9-dux1eef-liux1ec7u-cuxf3-sux1eb5n}

Trong R, chúng ta có các bộ dữ liệu mẫu, rất hữu ích để kiểm tra các hàm
và thao tác mới trước khi chuyển sang các bộ dữ liệu thực nghiệm.

\begin{Shaded}
\begin{Highlighting}[]
\CommentTok{\# Hiển thị các bộ dữ liệu có sẵn}
\FunctionTok{data}\NormalTok{()}

\CommentTok{\# Tải bộ dữ liệu iris}
\FunctionTok{data}\NormalTok{(iris)}

\CommentTok{\# Xem một số thông tin về bộ dữ liệu}
\FunctionTok{head}\NormalTok{(iris)}
\end{Highlighting}
\end{Shaded}

\begin{verbatim}
##   Sepal.Length Sepal.Width Petal.Length Petal.Width Species
## 1          5.1         3.5          1.4         0.2  setosa
## 2          4.9         3.0          1.4         0.2  setosa
## 3          4.7         3.2          1.3         0.2  setosa
## 4          4.6         3.1          1.5         0.2  setosa
## 5          5.0         3.6          1.4         0.2  setosa
## 6          5.4         3.9          1.7         0.4  setosa
\end{verbatim}

Ví dụ, bộ dữ liệu \texttt{iris} chứa thông tin về các đặc điểm của ba
loài hoa iris (setosa, versicolor, và virginica) với các phép đo về cánh
hoa (petal) và đài hoa (sepal).

\subsection{CÁC HÀM QUAN TRỌNG NHẤT ĐỂ KIỂM TRA DỮ
LIỆU}\label{cuxe1c-huxe0m-quan-trux1ecdng-nhux1ea5t-ux111ux1ec3-kiux1ec3m-tra-dux1eef-liux1ec7u}

\begin{Shaded}
\begin{Highlighting}[]
\CommentTok{\# head() {-} hiển thị một vài hàng đầu tiên của dữ liệu (mặc định là 6)}
\FunctionTok{head}\NormalTok{(iris)}
\end{Highlighting}
\end{Shaded}

\begin{verbatim}
##   Sepal.Length Sepal.Width Petal.Length Petal.Width Species
## 1          5.1         3.5          1.4         0.2  setosa
## 2          4.9         3.0          1.4         0.2  setosa
## 3          4.7         3.2          1.3         0.2  setosa
## 4          4.6         3.1          1.5         0.2  setosa
## 5          5.0         3.6          1.4         0.2  setosa
## 6          5.4         3.9          1.7         0.4  setosa
\end{verbatim}

\begin{Shaded}
\begin{Highlighting}[]
\CommentTok{\# tail() {-} hiển thị một vài hàng cuối cùng của dữ liệu (mặc định là 6)}
\FunctionTok{tail}\NormalTok{(iris)}
\end{Highlighting}
\end{Shaded}

\begin{verbatim}
##     Sepal.Length Sepal.Width Petal.Length Petal.Width   Species
## 145          6.7         3.3          5.7         2.5 virginica
## 146          6.7         3.0          5.2         2.3 virginica
## 147          6.3         2.5          5.0         1.9 virginica
## 148          6.5         3.0          5.2         2.0 virginica
## 149          6.2         3.4          5.4         2.3 virginica
## 150          5.9         3.0          5.1         1.8 virginica
\end{verbatim}

\begin{Shaded}
\begin{Highlighting}[]
\CommentTok{\# str() {-} hiển thị cấu trúc dữ liệu (lớp của biến, số lượng quan sát, v.v.)}
\FunctionTok{str}\NormalTok{(iris)}
\end{Highlighting}
\end{Shaded}

\begin{verbatim}
## 'data.frame':    150 obs. of  5 variables:
##  $ Sepal.Length: num  5.1 4.9 4.7 4.6 5 5.4 4.6 5 4.4 4.9 ...
##  $ Sepal.Width : num  3.5 3 3.2 3.1 3.6 3.9 3.4 3.4 2.9 3.1 ...
##  $ Petal.Length: num  1.4 1.4 1.3 1.5 1.4 1.7 1.4 1.5 1.4 1.5 ...
##  $ Petal.Width : num  0.2 0.2 0.2 0.2 0.2 0.4 0.3 0.2 0.2 0.1 ...
##  $ Species     : Factor w/ 3 levels "setosa","versicolor",..: 1 1 1 1 1 1 1 1 1 1 ...
\end{verbatim}

\begin{Shaded}
\begin{Highlighting}[]
\CommentTok{\# summary() {-} tóm tắt dữ liệu trong tập (thống kê mô tả, v.v.)}
\FunctionTok{summary}\NormalTok{(iris)}
\end{Highlighting}
\end{Shaded}

\begin{verbatim}
##   Sepal.Length    Sepal.Width     Petal.Length    Petal.Width   
##  Min.   :4.300   Min.   :2.000   Min.   :1.000   Min.   :0.100  
##  1st Qu.:5.100   1st Qu.:2.800   1st Qu.:1.600   1st Qu.:0.300  
##  Median :5.800   Median :3.000   Median :4.350   Median :1.300  
##  Mean   :5.843   Mean   :3.057   Mean   :3.758   Mean   :1.199  
##  3rd Qu.:6.400   3rd Qu.:3.300   3rd Qu.:5.100   3rd Qu.:1.800  
##  Max.   :7.900   Max.   :4.400   Max.   :6.900   Max.   :2.500  
##        Species  
##  setosa    :50  
##  versicolor:50  
##  virginica :50  
##                 
##                 
## 
\end{verbatim}

\begin{Shaded}
\begin{Highlighting}[]
\CommentTok{\# Các hàm hữu ích khác}
\FunctionTok{dim}\NormalTok{(iris)  }\CommentTok{\# kích thước của khung dữ liệu (số hàng, số cột)}
\end{Highlighting}
\end{Shaded}

\begin{verbatim}
## [1] 150   5
\end{verbatim}

\begin{Shaded}
\begin{Highlighting}[]
\FunctionTok{names}\NormalTok{(iris)  }\CommentTok{\# tên của các cột}
\end{Highlighting}
\end{Shaded}

\begin{verbatim}
## [1] "Sepal.Length" "Sepal.Width"  "Petal.Length" "Petal.Width"  "Species"
\end{verbatim}

\begin{Shaded}
\begin{Highlighting}[]
\FunctionTok{class}\NormalTok{(iris)  }\CommentTok{\# lớp của đối tượng}
\end{Highlighting}
\end{Shaded}

\begin{verbatim}
## [1] "data.frame"
\end{verbatim}

\subsection{CÁC THAO TÁC CƠ BẢN VỚI KHUNG DỮ
LIỆU}\label{cuxe1c-thao-tuxe1c-cux1a1-bux1ea3n-vux1edbi-khung-dux1eef-liux1ec7u}

\begin{Shaded}
\begin{Highlighting}[]
\CommentTok{\# Chọn một tập con của dữ liệu}
\NormalTok{small\_iris }\OtherTok{\textless{}{-}}\NormalTok{ iris[}\DecValTok{1}\SpecialCharTok{:}\DecValTok{10}\NormalTok{, }\FunctionTok{c}\NormalTok{(}\StringTok{"Sepal.Length"}\NormalTok{, }\StringTok{"Species"}\NormalTok{)]}
\NormalTok{small\_iris}
\end{Highlighting}
\end{Shaded}

\begin{verbatim}
##    Sepal.Length Species
## 1           5.1  setosa
## 2           4.9  setosa
## 3           4.7  setosa
## 4           4.6  setosa
## 5           5.0  setosa
## 6           5.4  setosa
## 7           4.6  setosa
## 8           5.0  setosa
## 9           4.4  setosa
## 10          4.9  setosa
\end{verbatim}

\begin{Shaded}
\begin{Highlighting}[]
\CommentTok{\# Thêm một cột mới}
\NormalTok{small\_iris}\SpecialCharTok{$}\NormalTok{New\_Column }\OtherTok{\textless{}{-}} \FunctionTok{c}\NormalTok{(}\DecValTok{1}\SpecialCharTok{:}\DecValTok{10}\NormalTok{)}
\NormalTok{small\_iris}
\end{Highlighting}
\end{Shaded}

\begin{verbatim}
##    Sepal.Length Species New_Column
## 1           5.1  setosa          1
## 2           4.9  setosa          2
## 3           4.7  setosa          3
## 4           4.6  setosa          4
## 5           5.0  setosa          5
## 6           5.4  setosa          6
## 7           4.6  setosa          7
## 8           5.0  setosa          8
## 9           4.4  setosa          9
## 10          4.9  setosa         10
\end{verbatim}

\begin{Shaded}
\begin{Highlighting}[]
\CommentTok{\# Lọc dữ liệu dựa trên điều kiện}
\NormalTok{setosa\_only }\OtherTok{\textless{}{-}}\NormalTok{ iris[iris}\SpecialCharTok{$}\NormalTok{Species }\SpecialCharTok{==} \StringTok{"setosa"}\NormalTok{, ]}
\FunctionTok{head}\NormalTok{(setosa\_only)}
\end{Highlighting}
\end{Shaded}

\begin{verbatim}
##   Sepal.Length Sepal.Width Petal.Length Petal.Width Species
## 1          5.1         3.5          1.4         0.2  setosa
## 2          4.9         3.0          1.4         0.2  setosa
## 3          4.7         3.2          1.3         0.2  setosa
## 4          4.6         3.1          1.5         0.2  setosa
## 5          5.0         3.6          1.4         0.2  setosa
## 6          5.4         3.9          1.7         0.4  setosa
\end{verbatim}

\begin{Shaded}
\begin{Highlighting}[]
\CommentTok{\# Sắp xếp dữ liệu}
\NormalTok{sorted\_iris }\OtherTok{\textless{}{-}}\NormalTok{ iris[}\FunctionTok{order}\NormalTok{(iris}\SpecialCharTok{$}\NormalTok{Sepal.Length), ]}
\FunctionTok{head}\NormalTok{(sorted\_iris)}
\end{Highlighting}
\end{Shaded}

\begin{verbatim}
##    Sepal.Length Sepal.Width Petal.Length Petal.Width Species
## 14          4.3         3.0          1.1         0.1  setosa
## 9           4.4         2.9          1.4         0.2  setosa
## 39          4.4         3.0          1.3         0.2  setosa
## 43          4.4         3.2          1.3         0.2  setosa
## 42          4.5         2.3          1.3         0.3  setosa
## 4           4.6         3.1          1.5         0.2  setosa
\end{verbatim}

\begin{Shaded}
\begin{Highlighting}[]
\CommentTok{\# Tóm tắt theo nhóm}
\FunctionTok{aggregate}\NormalTok{(Sepal.Length }\SpecialCharTok{\textasciitilde{}}\NormalTok{ Species, }\AttributeTok{data =}\NormalTok{ iris, }\AttributeTok{FUN =}\NormalTok{ mean)}
\end{Highlighting}
\end{Shaded}

\begin{verbatim}
##      Species Sepal.Length
## 1     setosa        5.006
## 2 versicolor        5.936
## 3  virginica        6.588
\end{verbatim}

\subsection{LƯU KHUNG DỮ LIỆU}\label{lux1b0u-khung-dux1eef-liux1ec7u}

\begin{Shaded}
\begin{Highlighting}[]
\CommentTok{\# Lưu khung dữ liệu vào một tệp CSV}
\FunctionTok{write.csv}\NormalTok{(iris, }\StringTok{"iris\_data.csv"}\NormalTok{, }\AttributeTok{row.names =} \ConstantTok{FALSE}\NormalTok{)}

\CommentTok{\# Lưu khung dữ liệu vào một tệp RDS (định dạng R)}
\FunctionTok{saveRDS}\NormalTok{(iris, }\StringTok{"iris\_data.rds"}\NormalTok{)}

\CommentTok{\# Lưu nhiều đối tượng vào một tệp RData}
\FunctionTok{save}\NormalTok{(iris, mtcars, }\AttributeTok{file =} \StringTok{"multiple\_datasets.RData"}\NormalTok{)}
\end{Highlighting}
\end{Shaded}

\subsection{KẾT LUẬN}\label{kux1ebft-luux1eadn}

Khung dữ liệu (data frame) là cấu trúc dữ liệu cơ bản và quan trọng nhất
trong R, đặc biệt trong phân tích dữ liệu và học máy. Chúng cung cấp một
cách linh hoạt để lưu trữ và thao tác với dữ liệu dạng bảng, với các cột
có thể có các kiểu dữ liệu khác nhau.

Trong phần tiếp theo, chúng ta sẽ tìm hiểu về các kỹ thuật nâng cao hơn
để thao tác với khung dữ liệu, bao gồm việc sử dụng các gói trong
tidyverse như dplyr để xử lý dữ liệu hiệu quả hơn.

\end{document}
